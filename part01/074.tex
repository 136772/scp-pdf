\chapter[SCP-074 量子土鳖]{
    SCP-074 Quantum Woodlouse\\
    SCP-074 量子土鳖
}

\label{chap:SCP-074}

\bb{项目编号:}SCP-074

\bb{项目等级:}Euclid

\bb{文件已更新}

\bb{特殊收容措施:}SCP-074位于Site-81。SCP-074可归类为一种动态信息危害,目前尚无任何评估其异常属性的研究报告。曾经研究过SCP-074并持有物理学学士或更高学位的研究员,以及访问过编号074-317E档案的人士,必须与SCP-074保持至少半径5km以上的空间距离。

SCP-074被收容在一个大小为6m x 6m x 3m\footnote{容器体积若是小于这个范围,会增加该SCP收容失效事件自然发生的概率。}的强化玻璃室内,该容器能滤除所有紫外线。在其外部,是一座由单色安全灯照明且没有窗户的房间,相当于第二层收容措施。容器的表面由培养在支架上的一张张人类皮肤组织薄膜交错排列,层叠包裹,其覆盖率需达到95\%。这些人皮薄膜至少得有3(三)毫米厚,温度要维持在37(三十七)摄氏度。皮肤组织的供体必须是D级人员,其受教育水平要在初中至高中之间,不能低,也不能高。因为SCP-074-1的缘故,每天都要对所有皮膜进行基础检查;一经发现SCP-074-1,立即将其切除并焚化。

每天,需将新鲜的苹果(或苹果属植物)树叶、树皮、果实切碎后,通过机械配送装置喂给SCP-074。这些“食物”必须是水培法栽种,以防止外来生物或是其他污染物。

如果收容失效事件自然发生,工作人员可以通过一定方法强制SCP-074回归容器。首先,迅速将四颗完整的生苹果分别塞入SCP-074的四套下颚,然后直接用物理手段推动SCP-074朝正确的方向前进,并张开手掌轻拍它的复眼,或者用0.5%甲酸溶液喷洒其前端那对触须。

\bb{描述:}SCP-074是一种异常有机体,能够宏观上地使用多种量子性质,从而修改其周身一定空间范围内的物理定律。这些内容会传达给处在SCP-074附近的人,令他们精确获悉所修改的法则原理。曾有研究人员想要弄清为何SCP-074具有这种修改的能力,事实上那些研究出来的结果都是SCP-074运用自身能力修改所得。编号074-317E档案——所有与SCP-074有关的,不论是过去还是现在的反常物理现象皆立案在内——允许3级以上成员观看;访问过此份档案相关人员必须调离SCP-074项目部门,不得再以任何理由接近Site-81周边5km范围内。

已经多次发现,SCP-074具有传送自身的能力,其出现的位置通常在强化玻璃室外周边3米内;这可以被认为,或者比拟为——“量子隧道穿越”。

基金会的昆虫学家已经初步确定了SCP-074属于一种等足目甲壳动物,俗称“土鳖”。其惯性质量约1700千克,但引力质量只有约375克。据估计其体积为1.7立方米,约是一辆紧凑型轿车的大小。

SCP-074的个体都是雌性\footnote{尽管这不是一种典型的具备囊袋或“育儿袋”的等足目动物,但它是卵生的。},繁衍方式为孤雌生殖;在其散发着冷光源的产卵器尖端,存在一种球形器官,能周期性地\footnote{屏蔽所有紫外线时,SCP-074的繁殖率是每小时1.3倍;若是暴露在自然光下,其繁殖速率将达到每小时29.2倍。}吐出一种具备非电离辐射形式的“相干波包”\footnote{能够正确理解“波包”概念的人必须调离SCP-074项目部门。},这可能就是SCP-074自体受精产生的“具现化”\footnote{“具现化”就是字面意思上的出现了实体化的情况。}之卵(以下简称SCP-074-1)。SCP-074-1的实体会优先具现化,从而接触到人类血肉,并汲取该生物的物理知识\footnote{只要是物理知识,哪怕再基本再低级,甚至是初等学历的人,只要是其脑内存在的科技文明概念,例如——磁铁能吸引或排斥对方,物质是由原子构成的,光存在一定速度——都会被SCP-074-1汲取利用。}。在没有合适的人类宿主供其寄生的情况下,这种波包会在其他生物有机体内具现化,甚至是无机物内;然而,若没有达到适当的孵化条件,这些卵就会枯萎死亡,在宏观尺度上能观察到其所留下的辐射损伤痕迹。这些波包会随着时间衰减,而且一旦超过SCP-074周身400m,便没再被观察到波包或者是这种损伤痕迹的情况。成功孵化出SCP-074-1实体的概率取决于宿主曝光在紫外线下的程度:平均每天曝光30分钟,持续一个月,便能观察到SCP-074-1实体从2毫克成长到8千克\footnote{实际上这种称量结果是在用外科手术切除并杀死了SCP-074-1之后进行的。};反之,将自然光隔离一个月,三个同时被观察的实体最终能切下来的部分仅有600g,680g以及710g。有关SCP-074-1的具体大小变化以及如何依附宿主的发育史、生命周期,目前尚未有一个完整的答案。
