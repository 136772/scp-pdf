\chapter[SCP-080 “鬼”]{
    SCP-080 Dark Form\\
    SCP-080 “鬼”
}

\label{chap:SCP-080}

\bb{项目编号:}SCP-080

\bb{项目分类:}Euclid

\bb{特殊收容措施:}SCP-080被收容在一间4米*4米的房间里,房间南侧有一个小前厅供研究人员进入。北侧连接着一个观察室,观察室和主房间之间的窗户拉着不透光的窗帘,只有当观察室的门是关着时才能拉开窗帘。无论如何不得尝试将SCP-080移出此收容间。收容室无论何时都最多只能用一(1)个不超过7瓦的白炽灯泡来照明。任何可以发出亮光的设备也绝对不可以带到收容室来。任何可以用来覆盖、遮掩或其他任何可以隐藏SCP-080的物品都绝对不能带入SCP-080的收容室。未遵守上述规定的将会被惩罚。

\bb{描述:}目前未知SCP-080是否有一个物质的形体或“身体”,因为任何试图与SCP-080的物理上互动都失败了,并且遇到了不良的后果(参见\hyperref[sec:DOC-experiment-log-080-2]{实验日志 080-2})。研究人员和工作人员对他们所看见的SCP-080的描述各不相同,从影子到人形雕像都有,但它们唯一的共同点是有两个冒烟的“眼睛”。需要所有研究员注意的是, SCP-080会诱使任何进入其收容间的人产生一种无法抑制的睡意。大约30分钟后,所有观测它的人都会被强迫进入快速眼动睡眠并且受到不可逆的精神伤害。即使从另一个房间观测SCP-080也会发生上述状况(参见事故-080-1)。不论何时,若当SCP-080找到了把他自己“藏起来”的办法(例如藏在一个橱柜里、床下、盖着一条床单等),它会彻底消失。此外,如果任何亮度强过一个普通的孩童用夜灯的光照进SCP-080的收容间的话,SCP-080会立刻消失。这两种情况任其一种发生都会被认定为收容失效,且任何对此负责的工作人员都将会被严厉地惩罚并可能会被重新分配。如需了解SCP-080的起源以及它被基金会收容的过程,请查看附加文档{[}删除]。

\bb{附录080-A:}在20██年██月██日,█████博士和他的直属部下开始抱怨他们的令人不安的梦越来越多。鉴于到目前为止他与SCP-080的互动最多,推测SCP-080或许有即使在没有直接被观测的情况下仍可以对附近的人员造成某种模因影响的能力。

\bb{附录080-B:}从20██年██月██日起,在█████博士自杀之后,所有在SCP-080部门的工作人员被命令记录下自己的梦境,并当梦境不可控地变得越来越暴力化或者恐怖化时必须通知站点的心理学家。

\bb{附录080-C:}所有工作人员需注意,若观察室的门未关上则绝不可拉开不透光的窗帘,且不论什么原因暴露时间都不得超过三十(30)分钟(参见事故-080-1)。

\bb{附录080-D:}高级研究员注意到,一些研究员无法看到SCP-080,且对其的影响完全免疫。任何觉得自己无法看到SCP-080的工作人员请向███████博士报告以进行规定的测试。

\bb{附录080-E:}在事故080-1发生之后,建议每月向SCP-080“喂”一名D级人员来中和SCP-080对站点工作人员的精神影响。(O5-█:批准)

\begin{scpbox}

\bb{事故080-1:}在20██年██月██日,两名高级研究员进行了一项计划好的观察。研究员进入了观察室并且拉开了窗帘。他们认为观察室像过去一样足以保护他们免受SCP-080的影响,于是开始了观察。观察开始后大约40分钟,两位研究员都突然昏睡过去。回收他们之后,{[}数据删除]。

(值得注意的是,在此事件发生之后,所有在工作区域内的有恶梦或压抑症状的人都报告了晚上能安稳入眠而且第二天普遍心情较佳。)

\end{scpbox}

\bb{笔记:}\ii{全体员工不得再把SCP-080叫做“鬼(The Boogieman)”。}-██████博士

\chapter{实验日志080-2}

\label{chap:DOC-experiment-log-080-2}

\bb{实验080-2-A:}

\bb{日期:}20██年█月█日\\
\bb{测试者:}D-080-1,男性,19岁\\
\bb{程序:}测试者被送进SCP-080的收容间\\
\bb{细节:}

\ii{测试者在14点26分进入SCP-080的收容间。}

\begin{scpbox}

\bb{<记录开始,{[}14:26]>}

█████博士:D-080-1,能看到什么吗?

D-080-1:不。这儿黑得不行。

\ii{(测试者沉默了几分钟)}

D-080-1:你刚刚是不是放什么东西进来了?我感觉有东西在看着我。

█████博士:没,D-080-1。我没放任何东西进来。

D-080-1:这他妈的是什么?\ii{(观察到测试者绊倒在地上。)}

█████博士:你看到什么了?

D-080-1:有片巨大的黑影,就像有人就站在墙角。哦老天,它正看着我!让我出去!\ii{(D-080-1开始敲通往前厅的门。)}

█████博士:它在看着你?请描述下它的外貌。

D-080-1:我不知道!噢老天让我出去吧。我不想呆在这。\ii{(D-080-1开始啜泣。)}

█████博士:D-080-1,告诉我它的外貌特征。然后我们就放你出来。

D-080-1:它-它看起来像个人形,在角落里驼着背。

█████博士:是个人吗?

D-080-1:比人大得多。\ii{(D-080-1开始打哈欠。)}它仍在盯着我,博士,仍在盯着我。

█████博士:靠近它。

\ii{(D-080-1开始向房间角落移动,很明显快站不住了。)}

D-080-1:它的眼睛就像蒸汽一样,一直盯着我。好像想让我做什—\ii{(D-080-1再次倒在地上。)}

█████博士:D-080-1,能听到吗?

\ii{(测试者在随后的5分钟里无回应)}

\bb{<记录结束,{[}14:58]>}

\end{scpbox}

\ii{据推测测试者昏睡了过去。随后未发现D-080-1的尸体,推测SCP-080吃掉了D-080-1。}

\bb{实验080-2-B:}

\bb{日期:}20██年██月██日\\
\bb{测试者:}D-080-2,女性,30岁\\
\bb{程序:}测试者被送进SCP-080的收容间,试图物理接触SCP-080\\
\bb{细节:}

\ii{测试者在█████博士的指示下于17点35分进入SCP-080的收容间。}

\begin{scpbox}

\bb{<记录开始,{[}17:35]>}

█████博士:告诉我你看见了什么。

D-080-2:我屁都没看见。为啥这儿这么黑?

D-080-2:\ii{(几分钟后)}我艹,这是啥?它就立在房子中间。

█████博士:描述下你看到的。

D-080-2:它就站在房子正中间,我猜…我能看到一双眼睛。

█████博士:请你靠近它,告诉我们有什么变化。

D-080-2:我很困…你刚说什么来着?

█████博士:伸出手摸摸它。告诉我感觉如何。

D-080-2:你让我去摸那玩意?

█████博士:是的,请上前。

\ii{(在几分钟的争论后,推定D-080-2上前摸了SCP-080,此时开始测试者再没有任何回应。)}

\bb{<记录结束,{[}18:12]>}

\end{scpbox}

\ii{随后发现测试者在SCP-080的收容间的角落里睡着了,她似乎没有在实验中受到伤害。在医务人员检查了D-080-2并认为她的身体状况良好后,对测试者进行了问话。(请查阅附加询问记录080-1)}

\bb{实验080-3-C:}

\bb{时间:}20██年██月██日\\
\bb{测试者:}D-080-3,男性,24岁\\
\bb{程序:}测试者在被注射了大剂量安非他明后被送入SCP-080的收容间。

\ii{一进入收容间,测试者就被要求告诉研究员他看到了什么。测试者描述了在屋子的正中间有一个像影子一样的人形。测试者被告知站着不动,并将任何变化告诉研究员。实验开始十分钟后,测试者开始打哈欠且变得明显的害怕了起来。测试者变得不合作,试着逃离收容间。逃跑失败后,测试者宣称将伤害SCP-080,据推测接下来其试着攻击SCP-080。测试者刚一这么做就立刻倒下了。}

\ii{D-080-3的尸体随后被从SCP-080的收容间里回收,似乎犯了心脏病。在回收D-080-3的尸体时,研究员们反映有强烈的不安感和被监视着的感觉,并且更加强烈地意识到SCP-080处于房间中。}

\bb{询问记录080-1:}

\begin{scpbox}

被问话者:D-080-2

问话者:█████博士

前言:\ii{D-080-2在一场当其在SCP-080的收容间里呆了37分钟后仍未得到任何实验结果的实验后被问话。}

\bb{<记录开始>}

█████博士:请说出你在实验时的记忆。

D-080-2:\ii{(█████博士随后记下了D-080-2的表情恍惚,语调没有活力。)}你让我走过去摸它,我不想摸,我不想。

█████博士:你碰到它之后发生了什么?

D-080-2:发生了什么?你没看见?我摸了它啊。之后我动不了了,它盯着我,我就是动不了。

█████博士:你有几分钟没回话,当时出什么事了?

D-080-2:\ii{(变得越来越激动)}它就盯着我,我不能动也不能呼吸。\ii{(D-080-2开始过度吸气。)}

█████博士:冷静,D-080-2,深呼吸。你还记得你睡着了吗?

D-080-2:它就一直盯着我,它没动过但我感觉事情不对!我躺在地板上,它们到处都是。它们就在这!\ii{(D-080-2开始尖叫然后猛然地站了起来。)}不,它们想把我带走,带回那玩意那去。我不回去!你们拖不走我!

\ii{D-080-2猛地冲向█████博士,随后被处决。}

\bb{<记录结束>}

\end{scpbox}

