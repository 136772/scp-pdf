\chapter[SCP-080 “鬼”]{
    SCP-080 Dark Form\\
    SCP-080 “鬼”
}

\label{chap:SCP-080}

\bb{项目编号:}SCP-080

\bb{项目分类:}Euclid

\bb{特殊收容措施:}SCP-080被收容在一间4米*4米的房间里,房间南侧有一个小前厅供研究人员进入。北侧连接着一个观察室,观察室和主房间之间的窗户拉着不透光的窗帘,只有当观察室的门是关着时才能拉开窗帘。无论如何不得尝试将SCP-080移出此收容间。收容室无论何时都最多只能用一(1)个不超过7瓦的白炽灯泡来照明。任何可以发出亮光的设备也绝对不可以带到收容室来。任何可以用来覆盖、遮掩或其他任何可以隐藏SCP-080的物品都绝对不能带入SCP-080的收容室。未遵守上述规定的将会被惩罚。

\bb{描述:}目前未知SCP-080是否有一个物质的形体或“身体”,因为任何试图与SCP-080的物理上互动都失败了,并且遇到了不良的后果(参见\hyperref[chap:DOC-experiment-log-080-2]{实验日志 080-2})。研究人员和工作人员对他们所看见的SCP-080的描述各不相同,从影子到人形雕像都有,但它们唯一的共同点是有两个冒烟的“眼睛”。需要所有研究员注意的是, SCP-080会诱使任何进入其收容间的人产生一种无法抑制的睡意。大约30分钟后,所有观测它的人都会被强迫进入快速眼动睡眠并且受到不可逆的精神伤害。即使从另一个房间观测SCP-080也会发生上述状况(参见事故-080-1)。不论何时,若当SCP-080找到了把他自己“藏起来”的办法(例如藏在一个橱柜里、床下、盖着一条床单等),它会彻底消失。此外,如果任何亮度强过一个普通的孩童用夜灯的光照进SCP-080的收容间的话,SCP-080会立刻消失。这两种情况任其一种发生都会被认定为收容失效,且任何对此负责的工作人员都将会被严厉地惩罚并可能会被重新分配。如需了解SCP-080的起源以及它被基金会收容的过程,请查看附加文档{[}删除]。

\bb{附录080-A:}在20██年██月██日,█████博士和他的直属部下开始抱怨他们的令人不安的梦越来越多。鉴于到目前为止他与SCP-080的互动最多,推测SCP-080或许有即使在没有直接被观测的情况下仍可以对附近的人员造成某种模因影响的能力。

\bb{附录080-B:}从20██年██月██日起,在█████博士自杀之后,所有在SCP-080部门的工作人员被命令记录下自己的梦境,并当梦境不可控地变得越来越暴力化或者恐怖化时必须通知站点的心理学家。

\bb{附录080-C:}所有工作人员需注意,若观察室的门未关上则绝不可拉开不透光的窗帘,且不论什么原因暴露时间都不得超过三十(30)分钟(参见事故-080-1)。

\bb{附录080-D:}高级研究员注意到,一些研究员无法看到SCP-080,且对其的影响完全免疫。任何觉得自己无法看到SCP-080的工作人员请向███████博士报告以进行规定的测试。

\bb{附录080-E:}在事故080-1发生之后,建议每月向SCP-080“喂”一名D级人员来中和SCP-080对站点工作人员的精神影响。(O5-█:批准)

\begin{scpbox}

\bb{事故080-1:}在20██年██月██日,两名高级研究员进行了一项计划好的观察。研究员进入了观察室并且拉开了窗帘。他们认为观察室像过去一样足以保护他们免受SCP-080的影响,于是开始了观察。观察开始后大约40分钟,两位研究员都突然昏睡过去。回收他们之后,{[}数据删除]。

(值得注意的是,在此事件发生之后,所有在工作区域内的有恶梦或压抑症状的人都报告了晚上能安稳入眠而且第二天普遍心情较佳。)

\end{scpbox}

\bb{笔记:}\ii{全体员工不得再把SCP-080叫做“鬼(The Boogieman)”。}-██████博士
